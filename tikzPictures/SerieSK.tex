\begin{circuitikz}[scale=2, european, american inductors]
\ctikzset{bipoles/length=1.2cm}
	\draw
	(0,0)
		to[L=L, o-](1,0)
	(1,0) to[R=R](2,0)
	(2,0) to[C=C, -o](3,0);
\end{circuitikz}


% \begin{tikzpicture}[circuit ee IEC, x=2cm, y=2cm, semithick]
% 	\node (start) [contact] at (0,0) {};
% 	\node (end) [contact] at (3,0) {};
% 	\draw (start) to [inductor={info=L}] (1,0)
% 				  to [resistor={info=R}] (2,0) 
% 				  to [capacitor={info=C}](3,0)
% 				  -- (end);
% 	
% 	\node (placeholder1) at (0,0.5) {};
% 	\node (placeholder2) at (0,-0.5) {};	
% \end{tikzpicture}

% Die Nodes placeholder1 und placeholder2 sind nur dazu da, damit die ganze
% Abbildung wieder gleich hoch ist wie der Parallel-Schwingkreis (siehe
% ParallelSK.tex)
