\section{Schwingkreise}
\subsection{Freie Schwingung}
Die Werte $\textcolor{blue}{U_1,U_2,\beta_u}\text{ sowie
}\textcolor{red}{I_1,I_2,\beta_i}$ müssen aus den Anfangswerten bestimmt
werden.\\
\renewcommand{\arraystretch}{1.5}
\begin{tabular}{| p{4cm} | p{7cm} | p{7cm} |}
	\hline
		& \textbf{Parallelschwingkreis} 
		& \textbf{Serienschwingkreis} \\
	\hline
%		& \parbox{7cm}{\includegraphics[height=2cm]{../El4/bilder/lcrpara.jpg}}
%		& \parbox{7cm}{\includegraphics[height=2cm]{../El4/bilder/lcrseries.jpg}} \\
%	\hline	
	DGL &
	\begin{minipage}{7cm}
    	\vspace{0.1cm}
    	$\ddot{u} + \dfrac{1}{RC} \dot{u} + \dfrac{1}{LC} u = 0
    	\Leftrightarrow \ddot{u} + \dfrac{\omega_r}{Q_P} \dot{u} + \omega_r^2
    	u = 0$
    \end{minipage}& 
	\begin{minipage}{7cm}
    	\vspace{0.1cm}
    	$\ddot{i} + \dfrac{R}{L} \dot{i} + \dfrac{1}{LC} i = 0
    	\Leftrightarrow \ddot{i} + \dfrac{\omega_r}{Q_S} \dot{i} + \omega_r^2 i =
    	0$\\
    \end{minipage}\\
	\hline
	Resonanzfrequenz, Güte, Dämpfungsfaktor &
		\begin{minipage}{7cm}
        	\vspace{0.1cm}
       		$\omega_r = \frac{1}{\sqrt{LC}}\qquad Q_P = R\sqrt{\frac{C}{L}} =
       		\frac{R}{\omega_r L}=R\omega_rC =\frac{1}{2 \xi_P}$
       		\end{minipage}&
		\begin{minipage}{7cm}
        	\vspace{0.1cm}
	  		$\omega_r = \frac{1}{\sqrt{LC}}\qquad Q_S = \frac{1}{R}\sqrt{\frac{L}{C}}
	  		= \frac{\omega_r L}{R}=\frac{1}{R\omega_rC} = \frac{1}{2 \xi_S}$
	  		\end{minipage}\\

	\hline	
	\textcolor{darkgrey}{Standardstartbedingungen} &
	\begin{minipage}{7cm}
    	\vspace{0.1cm}
    	\textcolor{darkgrey}{$u(t=0)=U_0$\quad
    	$\dot{u}(t=0)=-\dfrac{U_o}{RC}$}\\
    \end{minipage} &
	\begin{minipage}{7cm}
     	\vspace{0.1cm}
    	\textcolor{darkgrey}{$i(t=0)=0$\quad
    	$\dot{i}(t=0)=\dfrac{U_o}{L}$}\\   
    \end{minipage}\\
	\hline
	Aperiodisch, $ Q < \frac{1}{2}$
		& $u(t) = \textcolor{blue}{U_1} e^{\alpha_1 t} + \textcolor{blue}{U_2} e^{\alpha_2 t}$ 
		& $i(t) = \textcolor{red}{I_1} e^{\alpha_1 t} + \textcolor{red}{I_2} e^{\alpha_2 t}$ \\
		& $\alpha_{1,2} = - \frac{\omega_r}{2 Q_P} \pm \omega_r \sqrt{\frac{1}{4 Q_P^2} - 1}$	
		& $\alpha_{1,2} = - \frac{\omega_r}{2 Q_S} \pm \omega_r \sqrt{\frac{1}{4Q_S^2} - 1}$\\
		& \textcolor{darkgrey}{$U_1 = U_0 \frac{\frac{\omega_r}{Q} +
		\alpha_2}{\alpha_2 - \alpha_1} \quad$ $U_2 = U_0 \frac{\frac{\omega_r}{Q} +
		\alpha_1}{\alpha_1 - \alpha_2}$}
		& \textcolor{darkgrey}{$I_1 = \frac{U_0}{(\alpha_1 - \alpha_2)L} \quad I_2
		= -I_1$}
		\vspace{0.1cm}
		\\
	\hline 	
	Kritisch, $ Q = \frac{1}{2}$
		& $u(t) = (\textcolor{blue}{U_1} + \textcolor{blue}{\beta_u} t) 
		e^{\alpha t} = (\textcolor{blue}{U_1} + \textcolor{blue}{\beta_u} t) 
		e^{- \omega_r t}$ 
		& $u(t) = (\textcolor{red}{I_1} + \textcolor{red}{\beta_i} t) 
		e^{\alpha t} = (\textcolor{red}{I_1} + \textcolor{red}{\beta_i} t) 
		e^{- \omega_r t}$ \\
% 		& $i(t) = \textcolor{red}{I_1} e^{- \omega_r t} +
% 		\textcolor{red}{\beta_i} t e^{- \omega_r t}$ \\ 
		& $\alpha_{1,2} = - \dfrac{\omega_r}{2 Q_P} = - \omega_r$ & 
		$\alpha_{1,2} = - \dfrac{\omega_r}{2 Q_S} = - \omega_r$\\
		& \textcolor{darkgrey}{$U_1 = U_0 \quad \beta = -U_0 \left(
		\frac{\omega_r}{Q}+\alpha \right)$} & 
		\textcolor{darkgrey}{$I_1 = 0 \quad \beta = \dfrac{U_0}{L}$}
		\vspace{0.1cm}
		\\
	\hline 	
	Periodisch, $ Q > \frac{1}{2}$
		& $u(t) = \textcolor{blue}{U_1} e^{\alpha_1 t} + 
					\textcolor{blue}{U_2} e^{\alpha_2 t} $
		& $i(t) = \textcolor{red}{I_1} e^{\alpha_1 t} + 
					\textcolor{red}{I_2} e^{\alpha_2 t} $ \\
% 	 	& $i(t) = \textcolor{red}{I_1} e^{-\frac{\omega_r}{2 Q_S} t} \cos{\omega_0 t} + 
% 					\textcolor{red}{I_2} e^{-\frac{\omega_r}{2 Q_S} t} \sin{\omega_0 t} $ \\
		& $\alpha_{1,2} = - \frac{\omega_r}{2 Q_P} \pm j \omega_r \sqrt{1 - \frac{1}{4 Q_P^2}}$	
		& $\alpha_{1,2} = - \frac{\omega_r}{2 Q_S} \pm j \omega_r \sqrt{1 - \frac{1}{4 Q_S^2}}$	\\
		& $\omega_0 = \omega_r \sqrt{1 - \frac{1}{4 Q_P^2}} \qquad \omega_0 \approx \omega_r (Q_P > 10)$ 
		& $\omega_0 = \omega_r \sqrt{1 - \frac{1}{4 Q_S^2}} \qquad \omega_0 \approx\omega_r (Q_S > 10)$\\
		& \textcolor{darkgrey}{$u(t) = \frac{U_0 \omega_r}{\omega_0} e^{- \frac{\omega_r}{2
					Q} t} \cos{[\omega_0 t + \arctan{\frac{1}{\sqrt{4 Q^2 -1}}}]}$}
		& \textcolor{darkgrey}{$I_1 =  \frac{U_0}{L \cdot 2 \cdot j \omega_0} \quad I_2 = -I_1$
		$ \quad i(t) = \frac{U_0}{\omega_0 L} e^{-\xi  \omega_r  t} \sin{(\omega_0 t)}$}
		%%\vspace{0.1cm}
		\\
	\hline
	\end{tabular}


\subsection{Erzwungene Schwingung}	
\begin{tabular}{| p{4cm} | p{14.4cm}|}
	\hline
	Maximalwerte &
	\begin{minipage}{7cm}
    	\vspace{0.1cm}
   		$I_{Lmax}=I_{Cmax}=\dfrac{I\cdot
   		Q_P}{\sqrt{1-\frac{1}{4Q_P}}}$ \vspace{0.1cm}\\ $\omega_{I_{Lmax}} =
   		\omega_r\sqrt{1-\frac{1}{2Q_P^2}}\qquad \omega_{I_{Cmax}} = \dfrac{\omega_r}{\sqrt{1-\frac{1}{2Q_P^2}}}$\\
   	\end{minipage}
	\vline \hspace{0.1cm}
	\begin{minipage}[b]{7cm}
    	Gilt auch für $U_{Lmax}=U_{Cmax}$ bei dem Serieschwingkreis, jedoch muss
    	$Q_P$ mit $Q_s$ ersetzt werden.
    	\vspace{0.2cm}    	
    \end{minipage}\\
	\hline
	Bandbreite, Verstimmung &
	\begin{minipage}{7cm}
		\includegraphics[width=6cm]{./bilder/Bandbreite.jpg}
   	\end{minipage}
	\vline \hspace{0.1cm}
	\begin{minipage}[c]{7cm}
  		$b_w = \omega_2 - \omega_1 = \dfrac{\omega_r}{Q}$\\
  		$B = \dfrac{f_r}{Q}=\dfrac{b_w}{2\pi}$\\\\
		$\eta = \dfrac{\omega}{\omega_r} - \dfrac{\omega_r}{\omega} \qquad
		\dfrac{Z}{R} = \dfrac{U}{U_{max}} = \dfrac{1}{\sqrt{1 + (\eta Q)^2}}$
\end{minipage}\\
	\hline
	Zeigerdiagramme &
	\begin{minipage}{4.6cm}
    	\vspace{0.1cm}
		\includegraphics[width=4cm]{./bilder/Zeigerdiagramm_w1.jpg}\\ 
    \end{minipage}
	\vline \hspace{0.1cm}
	\begin{minipage}{4.6cm}
  		\includegraphics[width=4cm]{./bilder/Zeigerdiagramm_wr.jpg}
    \end{minipage}
	\vline \hspace{0.1cm}
	\begin{minipage}{4.6cm}
 		\includegraphics[width=4cm]{./bilder/Zeigerdiagramm_w2.jpg}   
    \end{minipage}\\
	\hline
\end{tabular}
\renewcommand{\arraystretch}{1}

\subsection{Verlustbehafteter Schwingkreis}
Resonanzfrequenz $\omega_r$ tritt dort auf, wo 
$\operatorname{Im} \{ \underline{Y}(\omega) \} =
\operatorname{Im}\{\underline{Z}(\omega) \} = 0$.

\subsection{Resonanzkreis mit realen Elementen}
Umrechnungen Parallel- $\Longleftrightarrow$ Serieschaltungen von L \& R oder C \& R. \\

% Siehe http://www.causa-dura.ch/Scripts/aet\_skript.pdf, seite 121 
\renewcommand{\arraystretch}{1.1}
\begin{tabular}{| p{2cm} | p{8cm} | p{8cm} |}
	\hline
		& \textbf{Parallel $\Rightarrow$ Seriell}  
		& \textbf{Seriell $\Rightarrow$ Parallel} \\
	\hline
		L \& R
		& $ R_S = \dfrac{R_P (\omega L_P)^2}{R_P^2 + (\omega L_P)^2} \qquad 
			L_S = \dfrac{R_P^2 L_P}{R_P^2 + (\omega L_P)^2}  $
		& $ R_P = \dfrac{R_S^2 + (\omega L_S)^2}{R_S} \qquad 
			L_P = \dfrac{R_S^2 + (\omega L_S)^2}{\omega^2 L_S}   $ \\
	\hline	
		C \& R
		& $ R_S = \dfrac{R_P}{1 + (\omega R_P C_P)^2} \qquad 
			C_S = \dfrac{1 + (\omega R_P C_P)^2}{(\omega R_P)^2 C_P}$
		& $ R_P = \dfrac{(\omega C_S R_S)^2 + 1}{(\omega C_S)^2 R_S} \qquad
			C_P = \dfrac{C_S}{1 + (\omega C_S R_S)^2}$\\
	\hline
\end{tabular}
\renewcommand{\arraystretch}{1}
